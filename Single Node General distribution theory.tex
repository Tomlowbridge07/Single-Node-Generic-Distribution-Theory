\documentclass[a4paper,10pt]{article}

\usepackage{color}
\usepackage{xcolor}
\usepackage{tikz}
\usepackage{amsmath}
\usepackage{amssymb}
\usepackage{amsthm}
\usepackage{graphicx}
\usepackage{mathtools}
\usepackage{wrapfig}
\usepackage{multirow}
\usepackage{comment}
\usepackage{natbib}
%\usepackage{float}
\usepackage{appendix}
\usepackage{subfig}
\usepackage{enumitem}
\usepackage{newfloat}
\usepackage[utf8]{inputenc}
\usepackage{floatrow}

\usetikzlibrary{calc}
\usetikzlibrary{fit}
\usetikzlibrary{shapes.misc,calc, positioning, hobby, backgrounds}


%\DeclarePairedDelimiter{\floor}{\lfloor}{\rightfloor}
%\DeclarePairedDelimiter{\ceil}{\lceil}{\rceil}

\newcommand{\halflength}{\ensuremath{\floor{\frac{m}{2}}}}
\newcommand{\floor}[1]{\left \lfloor #1 \right \rfloor}
\newcommand{\ceil}[1]{\left \lceil #1 \right \rceil}

\newtheorem{theorem}{Theorem}[section]
\newtheorem{corollary}[theorem]{Corollary}
\newtheorem{lemma}[theorem]{Lemma}

\theoremstyle{definition}
\newtheorem{definition}[theorem]{Definition}

\theoremstyle{definition}
\newtheorem{example}[theorem]{Example}
 
\theoremstyle{remark}
\newtheorem*{remark}{Remark}

\theoremstyle{definition}
\newtheorem*{note}{Note}

\DeclareFloatingEnvironment[fileext=los,
    listname={List of Example Figures},
    name=Example Figure,
    placement=tbhp,
    within=section,]{examplefigure}

\title{Single Node general distribution theory}
\date{\today}
\author{Thomas Lowbridge \\ School of Mathematical Sciences \\ University of Nottingham}

\bibliographystyle{plain}

\begin{document}

\pagestyle{empty}
{
  \renewcommand{\thispagestyle}[1]{}
  \maketitle
  \tableofcontents  
}
\clearpage
\pagestyle{plain}


\setlength{\parindent}{0pt}
\setlength{\parskip}{1em}

\newpage
\pagenumbering{arabic}
\section{Introduction to Problem}
We have some attack time distribution at a single node, $X$, with support, $\Pi=[0,\pi]$, and $B=\ceil{\pi}$.We also have the room cap size, $b$, we also have a cost $c$ and arrival rate, $\lambda$.

This gives us the state space, $(s,v)$ for $0 \leq s \leq B+1$ and $0 \leq v \leq b$.

We will consider the Cost to Progress matrix, $\mathbf{C}=\mathbf{A}+\mathbf{O}$, where $\mathbf{A}$ is the cost to not act due to arrival process and $\mathbf{O}$ is the cost to not act due to the observed process.

\textbf{Some Basic Properties}
Because the arrival process is not dependent on the observed number, $A_{i,j}=A_{i,1}$, that is $\mathbf{A}$ can be represented by a vector $\mathbf{a}=a_{i}=A_{i,1}$.

If $b=0$,we have $\mathbf{O}=\mathbf{0}$ , then $\mathbf{C}=\mathbf{A}$.

Because the observed process is dependent on $s$ and $v$ just multiples the answer by $v$. We can summarize the matrix $\mathbf{O}$ by $\mathbf{o}=O_{i,2}$. We can also note this is only possible if $b \geq 1$.

Then we can redefine, $C_{i,j}=a_{i}+jo_{i}$.



By our current definition $a_{i}= c \lambda \int_{i-1}^{i} F_{X}(x) dx$ and $o_{i}= c (F_{X}(i)-F_{X}(i-1))$.  

So $a_{i}$ is increasing.

If we want to decide when $C_{i,j}$ is increasing (non-decreasing) in $i,j$, by definition it is non-decreasing in $j$. so we will only look at it from the point of non-decreasing in $i$.

If $o_{i}$ is non-decreasing then clearly it is non-decreasing.

However if $o_{i}$ is decreasing then we must look closer. If once $o_{i}$ becomes decreasing it has all $c_{i,j} \geq c \lambda$, then we can avoid the problem.

We could also restrict the problem to some $b=b^{*}$. For each decreasing row, we can restrict the column to only allow $c_{i,j} \geq c \lambda$.

We can define $b_{i}^{+}=\floor{\frac{a_{i}-c \lambda}{o_{i}}}$ and a set $\chi=\{i \, | \, o_{i}<0 \}$ and define our limit by $b^{*}=\min\limits_{i \in \chi} \{b_{i}^{+} \}$

Let us consider the difference in $c_{i,j}$'s , $c_{i,j}-c_{i-1,j}=a_{i}-a_{i}+j(o_{i}-o_{i-1})$

We can define $c_{i,j}=c \int_{i-1}^{i} \lambda F_{X}(x)+j f_{x}(x) dx$. We can say this is non-decreasing if $\lambda F_{X}(x)+ j f_{x}(x)$ is non-decreasing, i.e $\lambda f_{X}(x) + j f^{'}_{{X}}(x) \geq 0$.


We can equivalently solve $\frac{\lambda}{j}f_{X}(x) \geq |f_{X}'(x)|$ for a function which is not strictly increasing. This is done by Gronwell's inequality theorem to get use $f(x_{0})e^{-\frac{\lambda}{j}|x-x_{0}|} \leq f(x) \leq f(x_{0})e^{\frac{\lambda}{j}|x-x_{0}|}$ for some $x_{0}$ in the support of $f_{X}$. This is equivalent to say that the function is not increasing/decreasing exponentially.

If we only consider decreasing regions of $f_{X}$ can we just check those, well lets say that in the regions $[a_{i},b_{i}]$ for $i \in {1,...,n}$ for some $n$. The function is decreasing then we could apply the same idea of Gronwell's inequality and get $f(p)e^{-\frac{\lambda}{j}|x-p|} \leq f(x) \leq f(p)e^{\frac{\lambda}{j}|x-p|}$ for some chosen $p \in [a_{i},b_{i}]$ (all choices are valid, easy choice is $p=a_{i}$) for all $i \in n$.

Then this allows us to say that in the decreasing regions of $f_{X}$ the equation still holds, and in the other regions it is increasing so the equation holds. Hence it holds.

So we really only need the decreasing regions to be monitored for exponential decay, if they decrease too fast we can be in trouble.
  



%End of main part of document
\bibliography{mybib}

\newpage
\appendix
\pagenumbering{roman}
\appendixpage
\addappheadtotoc
\section{First appendix section}

\end{document}
